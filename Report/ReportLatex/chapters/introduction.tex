The purpose of this project is to explore and analyze a large dataset using two different database technologies. Our focus is on executing queries that a typical user would find insightful and engaging. To achieve this, we selected a dataset centered around recipes and their associated reviews, enriched with various attributes such as cooking time, descriptions, ratings, and more. This dataset not only offers a large amount of textual data but also provides opportunities to explore the relationships between different elements, making it suitable for showcasing the capabilities of our chosen database technologies.

We approached this project with the mindset of a typical university student, someone just like us, who might be looking for the best recipes to cook in different scenarios. We designed queries that align with practical, real-world questions, such as finding highly rated recipes for quick meals, discovering popular dishes for special occasions, or exploring connections between ingredients and user preferences.

For technologies, we selected Neo4j and Elasticsearch, adopting the unique strengths of both to maximize the potential of our dataset.

In particular, we chose Elasticsearch because it is well-suited for performing queries on textual data, making it an excellent choice for analyzing recipe descriptions, cooking instructions, and user reviews. Its advanced full-text search capabilities allow us to rank results based on relevance, enabling features like searching for recipes containing specific words or filtering reviews to find the most positive feedback. Furthermore, Elasticsearch’s scoring mechanisms provides a powerful way to prioritize results, making sure that users our queries return the most relevant and useful information. 

Neo4j, on the other hand, excels at handling complex relationships between data. With its graph-based structure, Neo4j allows us to dive deeper into the interconnected nature of our dataset. For example, we can explore relationships between recipes and their ingredients, map user preferences based on reviews, or identify clusters of similar recipes based on shared attributes. Neo4j is particularly effective for queries that involve traversing these connections, and its ability to visualize relationships makes it an invaluable tool for uncovering hidden patterns and insights within the dataset.

Together, these tools allowed us to create queries that solve real-world problems and meet the practical needs of a university student looking for the perfect recipe. By combining the strengths of both technologies, we were able to make the most of our dataset and provide users with useful and meaningful insights.