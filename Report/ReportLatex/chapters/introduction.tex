The purpose of this project is to explore and analyze a large dataset using two different database technologies. 
In order to better understand the necessity of potential stakeholders of such analysis, the focus has been centered around off-site students. In particular the topic of recipes has been outlined for its simplicity, universal scope and affiliation with the authors' interests.
To achieve this, a dataset centered around recipes and their associated reviews was selected, enriched with various attributes such as cooking time, descriptions, ratings, and more. This dataset not only offers a large amount of textual data but also provides opportunities to explore the relationships between different elements.

The project was approached with the mindset of a typical university student who might be looking for the best recipes to cook in different scenarios. Queries were designed to align with practical, real-world questions, such as finding highly rated recipes for quick meals, discovering popular dishes for special occasions, or exploring connections between ingredients and user preferences.

To uncover hidden patterns and insights within the dataset, exploring the relationships between various entities is a necessity. Investigating how recipes relate to their ingredients, understanding user preferences based on reviews, and identifying clusters of similar recipes based on shared attributes are some of the possibilities opened by this kind of analysis. Visualizing these connections is key to gaining a deeper understanding of the data. To achieve this, we needed a technology that excels at handling complex relationships and traversing data connections.

For these reasons, we chose \textbf{Neo4j}, a graph-based database that allows diving into the interconnected nature of the data set and performing advanced queries involving relationships.

Additionally, the data set is rich in textual data extracted from a natural language context which can become challenging for most of the db alternatives usable in this project. Luckily, there is a technology that thrives where other fail: \textbf{Elasticsearch}.
Recipe descriptions, cooking instructions, and user reviews can be analyzed using full-text searches, ranking results based on relevance.

The analysis would have been significantly more challenging using a traditional SQL approach. Neo4j simplifies the exploration of connections within the dataset by eliminating the need for complex joins. Meanwhile, Elasticsearch enhances SQL's capabilities by enabling advanced text analysis, unlocking new possibilities for working with textual data.

As an extra, a simple web app called "\textbf{Taste Trios}" has been developed to showcase this project's results and to experiment with the two technologies in a "real production environment".

All the project materials can be found in this  \href{https://github.com/PaoloGinefra/SMBUD_project_24}{\textbf{GitHub repository}}